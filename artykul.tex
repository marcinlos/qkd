\documentclass[10pt]{article}


\usepackage{csagh}

\usepackage[utf8]{inputenc}
\usepackage{lmodern} 
\usepackage[T1]{fontenc}
\usepackage{microtype}

\usepackage{url}

\usepackage{amssymb}
\usepackage{amsthm}
\usepackage{amsmath}
\usepackage{braket}
\usepackage[ruled]{algorithm2e}

\usepackage{polski}
\usepackage[polish]{babel}


\newtheorem{theorem}{Theorem}

\begin{document}
\begin{opening}

\title{Quantum Key Distribution}
\author[AGH University of Science and Technology, anna.jagodzinska91@gmail.com]{Anna Jagodzińska}

\begin{abstract}
  (Abstrakt)
\end{abstract}

\keywords{key distribution, quantum cryptography}

\end{opening}

\section{Problem dystrybucji klucza}

Problem dystrybucji klucza w kryptografii to protokół umożliwiający dwóm komunikującym się bytom
uzgodnienie wspólnego tajnego klucza, nie znanego przez nikogo z zewnątrz. Od początku historii
kryptografii dystrybucja klucza stanowiła jeden z najbardziej kłopotliwych praktycznych jej aspektów.
Niezależnie od siły szyfru, bezpieczeństwo klucza jest niezbędne do bezpieczeństwa komunikacji i danych
przez niego chronionych. Przed wynalezieniem kryptografii asymetrycznej do bezpiecznej komunikacji 
potrzebny był obydwu stronom tajny, uzgodniony wcześniej klucz, który musiał w jakiś sposób fizycznie 
zostać przekazany przed rozpoczęciem komunikacji. W czasie drugiej wojny światowej japońska marynarka
przykładowo używała tzw. książek kodowych. Rozwiązanie to miało poważne wady -- ryzyko przechwycenia
ich przez nieprzyjaciela było dość duże, a co za tym idzie, kody musiały być często zmieniane, co 
wymagało ponownego ich drukowania i rozprowadzenia. Szczególnie w trudnych warunkach wojennych,
bezpieczna dystrybucja klucza, od której niejednokrotnie zależało powodzenie operacji militarnych,
a więc życie żołnieży, była zadaniem niełatwym. 

Dzisiaj, jakkolwiek dysponujemy kryptografią asymetryczną, szyfry symetryczne wciąż są szeroko 
stosowane, m. in. ze względu na znacznie większą wydajność \cite{IntelAES, IntelSSL} -- szyfry 
symetryczne zazwyczaj wykorzystują proste operacje bitowe, podczas gdy np. RSA potrzebuje arytmetyki
modularnej \footnotemark. Stąd, kryptografia asymetryczna często wykorzystywana jest do zainicjowania
komunikacji i wygenerowania klucza (np. protokół Diffiego-Hellmana), który następnie jest używany do
szyfrowania symetrycznego. Jednym z najszerzej znanych przykładów takiego schematu jest protokół
SSL, który używa protokołu Diffiego-Hellmana w początkowej fazie komunikacji, by ustalić klucz,
który używany będzie do zapewnienia bezpieczeństwa pozostałej jego części. Po zakończonym sukcesem
procesie uwierzytelniania, do komunikacji używany jest szyfr symetryczny (najczęściej AES).

\footnotetext{
Cytowane materiały Intela pokazują niecałe 4000 operacji na sekundę przy 1024-bitowym kluczu RSA,
oraz czasy rzędu kilku cykli na bajt dla szyfru symetrycznego AES -- różnica to co najmniej 3 
rzędy wielkości.
}


\section{Szyfrowanie one-time pad}

Jednym z najbardziej atrakcyjnych potencjalnych przypadków użycia dla ewentualnego w pełni bezpiecznego
protokołu dystrybucji klucza jest z pewnością tzw. szyfr z kluczem jednorazowym (one-time pad). Jego
idea opiera się na wykorzystaniu klucza długości nie mniejszej niż sama szyfrowana wiadomość, i
wykonywaniu prostych operacji na fragmentach szyfrogramu i klucza. Przykład konkretnej implementacji:
tekstujawny i klucz przedstawiane są jako ciągi znaków z pewnego alfabetu alfabetu \(n\)-znakowego 
\(\mathcal{A}\), reprezentowanego przez liczby naturalne \(0,\ldots n-1\), tj. tekst jawny 
\(T=t_1\ldots t_m\), klucz \(K=k_1\ldots k_m\), zaś szyfrogram powstaje poprzez dodawanie odpowiadających
sobie znaków tekstu i klucza modulo \(n\), tj.
\[
S_i=\left(t_i\oplus k_i\right),\qquad a\oplus b \triangleq a+b \mod n
\]
Deszyfracja polega na odwróceniu tej operacji, tj. odejmowaniu kolejnych znaków klucza od znaków
szyfrogramu. Szyfr ten jest całkowicie bezpieczny w sensie teorii informacji -- szyfrogram nie zawiera
żadnej informacji o zaszyfrowanym tekście, poza maksymalną jego długością \cite{Shannon49}. Precyzyjnie
wyrazić to można przy użyciu pojęć teorii informacji -- entropii i entropii warunkowej. Wartości te 
stanowią miarę niepewności, informacji, które niesie ze sobą wystąpienie jakiegoś zdarzenia (w tym
przypadku, wystąpienie danego tekstu jawnego, lub szyfrogramu).
Dla dyskretnej zmiennej losowej \(X\), przyjmującej wartości \(x_1,\ldots,x_n\) z prawdopodobieństwami
\(p(x_1),\ldots,p(x_n)\), entropia zdefiniowana jest jako
\[
\mathcal{H}(X)=-\sum_{i=1}^n p(x_i)\log p(x_i)
\]
zaś dla drugiej zmiennej \(Y\), o wartościach \(y_1,\ldots,y_m\) entropia warunkowa, mierząca niepewność
co do wartości \(X\) gdy dana jest wartość \(Y\), wyraża się poprzez
\[
\mathcal{H}(X\mid Y)=-\sum_{i,j=1}^{n,m}p(x_i,y_j)\log \frac{p(x_i)}{p(x_i,y_j)}
\]
(dla \(p(x_i,y_j)=0\) w sumie przyjmujemy jako składnik \(0\)). Łatwo pokazać, że dla dowolnych \(X\),
\(Y\) jest \(\mathcal{H}(X\mid Y)\leq \mathcal{H}(X)\). Jeśli za \(X\) przyjmiemy zmienną losową
odpowiadającą rozkładowi tekstów jawnych, a za \(Y\) odpowiadających im szyfrogramów, intuicyjnie
\(\mathcal{H}(X\mid Y)\) mierzy jak wiele informacji o tekście jawnym daje znajomość szyfrogramu --
dla małych wartości daje dużo informacji (np. jeśli szyfrogramem byłby sam tekst jawny, z powyższego
wzoru otrzymamy entropię warunkową równą \(0\)), zaś dla dużych -- niewiele. Dowód bezpieczeństwa szyfru
one-time pad polega na pokazaniu, że dla w pełni losowego klucza, entropia warunkowa tekstu jawnego
względem szyfrogramu jest równa entropii samego tekstu jawnego, tj. 
\(\mathcal{H}(X\mid Y) = \mathcal{H}(X)\). Intuicyjnie, szyfrogram nie daje żadnej informacji o tekście
jawnym, ta ,,ukryta'' informacja jest w kluczu.

Powyższe rezultaty odnośnie szyfru z kluczem jednorazowym mają duże historyczne znaczenie teoretyczne.
Aby jednak szyfr był faktycznie bezpieczny, konieczne jest spełnienie szeregu dość restrykcyjnych
założeń.

\begin{itemize}
  \item Klucz musi być długości co najmniej takiej, jakiej jest tekst jawny
  \item Klucz musi być idealnie losowy
  \item Klucz może być wykorzystany tylko raz -- ponowne wykorzystanie klucza umożliwia kryptoanalizę
    opartą np. na niejednorodności rozkładu tekstu jawnego \footnotemark.
\end{itemize}

\footnotetext{
Przykładowo, w ekstremalnym przypadku gdy klucz jest jedną literą, każde wystąpienie danego znaku będzie
odpowiadać takiemu samemu znakowi. Pomijając oczywisty atak typu bruteforce, patrząc na szyfrogram
można z dużym prawdopodobieństwem stwierdzić, które jego znaki odpowiadają najczęściej występującym
w języku, w którym napisany został tekst jawny, porównując po prostu częstotliwość wystąpień.
}

Wymagania te sprawiają, że używanie szyfru z kluczem jednorazowym w praktyce jest dość kłopotliwe.
Szczególnie konieczność stosowania dużych, unikalnych kluczy, które obydwie strony komunikacji muszą
posiadać przed jej rozpoczęciem. Wymaga to wcześniejszego przygotowania -- gdyby istniał sposób
bezpiecznej transmisji klucza rozmiaru samej wiadomości bez możliwości przechwycenia go przez 
potencjalnych napastników, równie dobrze można by użyć takiego kanału do przekazania samej wiadomości.

Mimo tych problemów, szyfry z kluczem jednorazowym były stosunkowo popularne i często wykorzystywane
w sytuacjach, gdzie komunikacja nie była częsta, a niezbędny był bardzo wysoki stopień bezpieczeństwa,
np. przez sowieckich szpiegów w okresie zimnej wojny. Problemem była oczywiście dystrybucja kluczy
-- przechowywane były one na bardzo małych, łatwych do ukrycia kartkach papieru, często nasączonych
substancjami łatwopalnymi, by ułatwić natychmiastowe ich zniszczenie bez zostawienia śladów po użyciu.

Mimo to, trudno wyobrazić sobie używanie takiego systemu współcześnie, do szyfrowania komunikacji
,,na codzień''. Fizyczna dystrybucja kluczy zdaje się być zbyt niepraktycznym rozwiązaniem w sytuacji,
gdy komunikacja następuje pomiędzy wieloma parami uczestników, dynamicznie, i przesyłane są duże
ilości danych. Co więcej, zostało udowodnione \cite{Shannon49}, że każdy szyfr bezpieczy z punktu
widzenia teorii informacji ma podobnie kłopotliwe wymagania. W obliczu braku w pełni bezpiecznych,
szybkich i wygodnych protokołów dystrybucji klucza zmuszeni jesteśmy korzystać z szyfrów, których
bezpieczeństwo oparte jest jedynie o przypuszczalną \footnotemark trudność obliczeniową pewnych 
problemów algorytmicznych, jak faktoryzacja, czy obliczanie logarytmu dyskretnego.

\footnotetext{
Pomijając kwestię P vs NP, w przypadku której, jakkolwiek pozostaje niepewność, większość ekspertów
jest mocno przekonana, że równość nie zachodzi, w przypadku problemów najczęściej leżących u podstaw
algorytmów kryptograficznych -- faktoryzacji i obliczania logarytmu dyskretnego -- nie wiadomo nawet,
czy są NP-zupełne.
}

\nocite{*}

\bibliographystyle{cs-agh}
\bibliography{bibliography}

\end{document}
