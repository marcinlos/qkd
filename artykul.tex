\documentclass[10pt]{article}


\usepackage{csagh}

\usepackage[utf8]{inputenc}
\usepackage{lmodern} 
\usepackage[T1]{fontenc}
\usepackage{microtype}

\usepackage{url}

\usepackage{amssymb}
\usepackage{amsthm}
\usepackage{amsmath}
\usepackage{braket}
\usepackage[ruled]{algorithm2e}

\usepackage{polski}
\usepackage[polish]{babel}


\newtheorem{theorem}{Theorem}

\begin{document}
\begin{opening}

\title{Quantum Key Distribution}
\author[AGH University of Science and Technology, anna.jagodzinska91@gmail.com]{Anna Jagodzińska}

\begin{abstract}
  (Abstrakt)
\end{abstract}

\keywords{key distribution, quantum cryptography}

\end{opening}

\section{Problem dystrybucji klucza}

Problem dystrybucji klucza w kryptografii to protokół umożliwiający dwóm komunikującym się bytom
uzgodnienie wspólnego tajnego klucza, nie znanego przez nikogo z zewnątrz. Od początku historii
kryptografii dystrybucja klucza stanowiła jeden z najbardziej kłopotliwych praktycznych jej aspektów.
Niezależnie od siły szyfru, bezpieczeństwo klucza jest niezbędne do bezpieczeństwa komunikacji i danych
przez niego chronionych. Przed wynalezieniem kryptografii asymetrycznej do bezpiecznej komunikacji 
potrzebny był obydwu stronom tajny, uzgodniony wcześniej klucz, który musiał w jakiś sposób fizycznie 
zostać przekazany przed rozpoczęciem komunikacji. W czasie drugiej wojny światowej japońska marynarka
przykładowo używała tzw. książek kodowych. Rozwiązanie to miało poważne wady -- ryzyko przechwycenia
ich przez nieprzyjaciela było dość duże, a co za tym idzie, kody musiały być często zmieniane, co 
wymagało ponownego ich drukowania i rozprowadzenia. Szczególnie w trudnych warunkach wojennych,
bezpieczna dystrybucja klucza, od której niejednokrotnie zależało powodzenie operacji militarnych,
a więc życie żołnieży, była zadaniem niełatwym. 

Dzisiaj, jakkolwiek dysponujemy kryptografią asymetryczną, szyfry symetryczne wciąż są szeroko 
stosowane, m. in. ze względu na znacznie większą wydajność \cite{IntelAES, IntelSSL} -- szyfry 
symetryczne zazwyczaj wykorzystują proste operacje bitowe, podczas gdy np. RSA potrzebuje arytmetyki
modularnej \footnotemark. Stąd, kryptografia asymetryczna często wykorzystywana jest do zainicjowania
komunikacji i wygenerowania klucza (np. protokół Diffiego-Hellmana), który następnie jest używany do
szyfrowania symetrycznego. Jednym z najszerzej znanych przykładów takiego schematu jest protokół
SSL, który używa protokołu Diffiego-Hellmana w początkowej fazie komunikacji, by ustalić klucz,
który używany będzie do zapewnienia bezpieczeństwa pozostałej jego części. Po zakończonym sukcesem
procesie uwierzytelniania, do komunikacji używany jest szyfr symetryczny (najczęściej AES).

\footnotetext{
Cytowane materiały Intela pokazują niecałe 4000 operacji na sekundę przy 1024-bitowym kluczu RSA,
oraz czasy rzędu kilku cykli na bajt dla szyfru symetrycznego AES -- różnica to co najmniej 3 
rzędy wielkości.
}



\nocite{*}

\bibliographystyle{cs-agh}
\bibliography{bibliography}

\end{document}
